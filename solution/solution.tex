\documentclass{article} 

\usepackage[most]{tcolorbox}
\usepackage{xcolor}
\usepackage{enumerate}
\usepackage[a4paper, margin=1in]{geometry} 
\usepackage{titling}
\usepackage{amssymb}
\usepackage{lipsum}
\usepackage{mathtools}
\usepackage{amsthm} % For proof environments only, labelling of thms done with custom counter
\usepackage{amsmath}
\usepackage{amssymb}
\usepackage{textcomp} % TM logo, shits n giggles

%%% Misc commands %%%
\newcommand\iidsim{\stackrel{\mathclap{\tiny\mbox{i.i.d}}}{\sim}}
\newcommand\indist{\stackrel{\mathclap{\tiny\mbox{$\mathcal{D}$}}}{\longrightarrow}}
\newcommand\nidist{\stackrel{\mathclap{\tiny\mbox{$\mathcal{D}$}}}{\longleftarrow}}
\newcommand{\indep}{\perp\!\!\!\!\perp} 

\setlength{\parindent}{0pt} % Remove indentation upon new paragraph. 

%%% Title page information %%%
\title{Crossing Bridges}
\author{Jacob Green}
\date{\today}
\newcommand{\subtitle}{An investigation into a bridge crossing problem under funky constraints}
\newcommand{\institution}{Department of Mathematical Sciences, The University of Bath}
\newcommand{\keywords}{Logic, Discrete Optimisation}

\newcounter{definitioncount} % tcolorbox counter, for ease of self-reference
\newcounter{lemmacount}
\newcounter{examplecount}
\newcounter{theoremcount}
\newcounter{propositioncount}
\newcounter{corollarycount}
\newcounter{remarkcount}
\counterwithin{corollarycount}{section}
\counterwithin{propositioncount}{section}
\counterwithin{remarkcount}{section}
\counterwithin{theoremcount}{section}
\counterwithin{examplecount}{section}
\counterwithin{lemmacount}{section}
\counterwithin{definitioncount}{section}
\renewcommand{\thecorollarycount}{\thesection.\arabic{corollarycount}}
\renewcommand{\thepropositioncount}{\thesection.\arabic{propositioncount}}
\renewcommand{\theremarkcount}{\thesection.\arabic{remarkcount}}
\renewcommand{\thetheoremcount}{\thesection.\arabic{theoremcount}}
\renewcommand{\theexamplecount}{\thesection.\arabic{examplecount}}
\renewcommand{\thelemmacount}{\thesection.\arabic{lemmacount}}
\renewcommand{\thedefinitioncount}{\thesection.\arabic{definitioncount}}
\newcounter{claimcount} % Used for proof with lots of claims

\newtcolorbox{definition}[2][auto counter, number within=section]{
  colback=black!5!white, 
  colframe=black!50!white, 
  fonttitle=\bfseries, 
  coltitle=black,
  title=Definition \thedefinitioncount $\:$ (#2), 
  before upper = \refstepcounter{definitioncount},
  #1, % Optional params
}

\newtcolorbox{lemma}[2][auto counter, number within=section]{
  colback=blue!5!white, 
  colframe=blue!50!white, 
  fonttitle=\bfseries, 
  coltitle=black, 
  title=Lemma \thelemmacount $\:$ (#2),
  before upper = \refstepcounter{lemmacount} 
  #1, % Optional params
}

\newtcolorbox{example}[2][auto counter, number within=section]{
  colback=green!5!white, 
  colframe=green!50!white, 
  fonttitle=\bfseries, 
  coltitle=black, 
  title=Example \theexamplecount $\:$ (#2), 
  before upper = \refstepcounter{examplecount}
  #1, % Optional params
}

\newtcolorbox{problem}[2][auto counter, number within=section]{
  colback=purple!5!white, 
  colframe=purple!50!white, 
  fonttitle=\bfseries, 
  coltitle=black, 
  title=Problem $\:$ (#2), 
  #1, % Optional params
}

\newtcolorbox{theorem}[2][auto counter, number within=section]{
  colback=red!5!white,
  colframe=red!50!white, 
  fonttitle=\bfseries, 
  coltitle=black, 
  title=Theorem \thetheoremcount $\:$ (#2), 
  before upper = \refstepcounter{theoremcount},
  #1, % Optional params
}

\newtcolorbox{proposition}[2][auto counter, number within=section]{
  colback=purple!5!white,
  colframe=purple!50!white, 
  fonttitle=\bfseries, 
  coltitle=black, 
  title=Proposition \thepropositioncount $\:$ (#2), 
  before upper = \refstepcounter{propositioncount},
  #1, % Optional params
}

\newtcolorbox{corollary}[2][auto counter, number within=section]{
  colback=yellow!5!white,
  colframe=yellow!50!white, 
  fonttitle=\bfseries, 
  coltitle=black, 
  title=Corollary \thecorollarycount $\:$ (#2),
  before upper = \refstepcounter{corollarycount}, 
  #1, % Optional params
}

\newtcolorbox{remark}[2][auto counter, number within=section]{
  colback=black!5!white,
  colframe=black!50!white, 
  fonttitle=\bfseries, 
  coltitle=black, 
  title=Remark \theremarkcount $\:$ (#2),
  before upper = \refstepcounter{remarkcount}, 
  #1, % Optional params
}

\newtcolorbox[use counter=claimcount]{claim}[1][]{%
    colback=green!10, 
    colframe=green!50!black, 
    coltitle=black, 
    fonttitle=\bfseries, 
    boxrule=0.5mm, 
    colbacktitle=green!30,
    enhanced, % Allows additional options
    boxed title style={sharp corners}, 
    attach title to upper={},
    titlerule=0mm, 
    title=Claim: $\;$,
    before=\par\smallskip\noindent, 
    #1
}

\begin{document}

\begin{titlepage}
    \centering
    % Adding an image (optional)
    
    % Title
    {\Huge \bfseries \thetitle \par}
    \vspace{0.5cm}
    
    % Subtitle (if any)
    {\Large \subtitle \par}
    \vspace{1cm}
    
    % Author and institution
    {\large \theauthor \par}
    {\institution \par}
    \vspace{1cm}
    
    % Date
    {\large \thedate \par}
    \vspace{1.5cm}
    
    % Abstract section
    \begin{abstract}
        \lipsum[10]
    \end{abstract}
    \vspace{1cm}
    
    % Keywords section
    \textbf{Keywords:} \keywords
    \vfill % Pushes the following content to the bottom
    
    % Footer or further information
    \textit{}
\end{titlepage}

\newpage 

\setcounter{page}{1} % Start page numbering from 1 after title page

\section{Problem Statement $\&$ Cursory Remarks}

% Count over naturals
\setcounter{lemmacount}{1}
\setcounter{examplecount}{1}
\setcounter{theoremcount}{1}
\setcounter{propositioncount}{1}
\setcounter{corollarycount}{1}
\setcounter{remarkcount}{1}
\setcounter{definitioncount}{1}

\begin{problem}[]{$N$-crossing problem}
    Suppose $N$ people must cross a bridge. They may cross at most in pairs and they require a flashlight to cross, 
    of which there is only $1$. If a pair is to cross, they must walk at the speed of the slowest member. If person 
    $1 \leq i \leq N$ takes $2^i$ minutes to cross, what is the fastest time all $N$ people can cross the bridge? 
\end{problem}

To get a feel for the problem, lets solve the $N=3$ case. 

\begin{proof}[Solution for $3$-crossing problem]    
    Let the start of the bridge be $A$ and the end be $B$. Call a trip from $A \to B$ a {\it crossing} and a trip 
    from $B \to A$ a {\it return}. Then clearly there must be at least two crossings and one return. Also, we know 
    that in order to be optimal, crossings will always be done in pairs and returns will always be done alone. One 
    can argue that this is true rather simply (exercise 1). \\
    
    Assume for the moment the torch is always at $A$. Then the fastest way to get everyone across, noting we must 
    always cross in pairs, is $2^3 + 2^2 = 12$ minutes. The fastest we can return is in $2^1=2$ minutes, giving us 
    a lower bound of $12 + 2 = 14$ minutes. \\

    We now seek a way to cross achieving this lower bound. If person $1$ and $2$ cross first, person $1$ returns and
    person $1$ and $3$ cross this will take a total of $2^2 + 2^1 + 2^3 = 14$ minutes. Hence $14$ minutes is the 
    fastest achievable time.
\end{proof}

The structure here was rather intuitive. We select the fastest person to be the ``torch bearer'', i.e. the one who 
carries out the return trip, and then we are forced to be in optimal conditions (note $1$ and $3$ crossing first 
works out the exact same). Lets see if this logic prevails in the $N=4$ case. 

\begin{proof}[Solution to the $4$-crossing problem]
    One may argue, as before, that we need at least $3$ crossings and $2$ returns. The fastest way to get everyone 
    across, in exactly $3$ crossings (noting we always cross in pairs), is $2^4 + 2^2 + 2^2 = 22$ minutes. The fastest way to do $2$ returns 
    is $2^1 + 2^1 = 4$ minutes. So we have a lower bound of $28$ minutes. \\
    
    Unlike last time, this lower bound cannot be achieved. To see this, note in order to have a total return time of 
    $4$ minutes, we would need person $1$ to return twice. This forces our $3$ crossings to be person $1$ with 
    person $i$, $2 \leq i \leq 4$, in some order. If this is the case, we take a total of $2^2 + 2^3 + 2^4 + 2^1 
    + 2^1 = 32$ minutes. Contradiction! \\

    We know our total time must be even, as all individual crossings/returns are of even length. Thus the next 
    possible best bound is $30$. This can be acheived as follows: 
    \begin{enumerate}
        \item Person $1$ and $2$ cross together.
        \item Person $1$ returns. 
        \item Person $3$ and $4$ cross together. 
        \item Person $2$ returns.
        \item Person $1$ and $2$ cross together.
    \end{enumerate}
    This gives a total time of $2^2 + 2^1 + 2^4 + 2^2 + 2^2 = 30$ minutes, and hence this is our best possible time.
\end{proof}

The logic almost prevails. We have the fastest $2$ as the torch bearers, but we don't let the fastest bear the torch 
twice. \\

One thing we observed was that, in order to minimise time spent crossing, people would always cross in pairs and return alone. This is due to 
the simple reasoning that crossing alone requires one person to go back with the torch, and thus the number of people at side $B$ would remain 
constant but we'd increase our total time. Equivalent logic can be used to argue for solo returns. While a trivial observation, this will be key 
in solving the full problem. \\ 

This time, it was also optimal to pair up the slowest two walkers. This, as one can easily verify, is not true in the $N=3$ case. With similar 
arguing one can find the minimum in the $N=5$ case to be $56$, which is also obtained by pairing up the slowest two walkers. 

\section{Formal Statement $\&$ Lemmas}

Now let us formalise this problem. Let $\mathcal{P}_k^=(S)$ denote the subsets size $k$ of a set $S$, and $\mathcal{P}_{k}^\leq(S)$ the subsets 
of size at most $k$. Let $[N] := \{1, \dots, N\}$. \\

We know in order to be optimal, we will always be crossing in pairs and returning alone (except in the special case when a crossing gets everyone 
to $B$, then no one will return). Thus we may consider a strategy of getting everyone across as a sequence of {\it moves} ${\bf \omega} 
= (\omega_1, \omega_2) \in \mathcal{P}_2^=([N]) \times \mathcal{P}_1^\leq([N])$, noting the order we choose the two people in our crossing pair 
doesn't affect the crossing time and we have the possibility of no one returning. We say $\omega_1$ is the {\it crossing pair} and $\omega_2$ the 
{\it returner} for a move ${\bf \omega}$. \\

Let us also track the number of people at $B$ after a given sequence of moves. Let $f: [N] \to \{0,1\}$ map $f(i \; ; \; \omega_1, \dots \omega_k) 
= {\bf 1}\{\text{person } i \text{ is at } B\}$ and set $F(\omega_1, \dots, \omega_k) = \sum_{i=1}^N f(i \; ; \; \omega_1, \dots, \omega_k)$. We 
will call a sequence of moves $\Omega = (\omega_1, \dots, \omega_k)$ {\it complete} if $F(\Omega) = N$. 


\begin{lemma}[]{optimal number of moves}
  If $\Omega$ is complete, then $|\Omega| = N-1$. 
\end{lemma}

\begin{proof}
  Let $\Omega$ be a complete sequence of moves and let $\Omega_k$ denote the first $k$ of these moves. Then 
  \[F(\Omega_k) < N-2 \Longrightarrow F(\Omega_{k+1}) = 1 + F(\Omega_k)\]
  and 
  \[F(\Omega_k) = N-2 \Longrightarrow F(\Omega_{k+1}) = 2 + F(\Omega_k)\]
  as each move will leave one new person at side $B$ of the bridge, unless there are exactly two people remaining in which case we can send 
  both people over and make no return trip. Thus, setting $\Omega_0 = \emptyset$ so that $F(\Omega_0) = 0$, we have $F(\Omega_{N-1}) = N$ and 
  hence $|\Omega| = N-1$. 
\end{proof}




\end{document}